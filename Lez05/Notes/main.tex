\documentclass[10pt]{article}
\usepackage{amsmath}
\usepackage{amssymb}
\usepackage{anysize}
\usepackage{listings}
\usepackage{graphicx}
\usepackage{xcolor}


\definecolor{Blue}{rgb}{0.2,0.2,0.9}
\definecolor{Green}{rgb}{0,0.6,0}
\definecolor{Gray}{rgb}{0.5,0.5,0.5}
\definecolor{Purple}{rgb}{0.58,0,0.82}
\definecolor{background}{rgb}{0.95,0.95,0.92}

\definecolor{bashtext}{rgb}{0.9,0.9,0.9}
\definecolor{bashbackground}{rgb}{0.3,0.3,0.3}

\lstdefinestyle{mystyle}{
    backgroundcolor=\color{background},   
    commentstyle=\color{Green},
    keywordstyle=\color{Blue},
    numberstyle=\tiny\color{Gray},
    stringstyle=\color{Purple},
    basicstyle=\ttfamily\footnotesize,
    breakatwhitespace=false,         
    breaklines=true,                 
    captionpos=b,                    
    keepspaces=true,                 
    numbers=left,                    
    numbersep=5pt,                  
    showspaces=false,                
    showstringspaces=false,
    showtabs=false,                  
    tabsize=2
}
\lstset{style=mystyle}

\newcommand{\ttt}{\texttt}

\newcommand{\tcpp}[1]{\hspace{10pt}\colorbox{background}{\textcolor{black}{\texttt{#1}}}}
\newcommand{\tbash}[1]{\hspace{10pt}\colorbox{bashbackground}{\textcolor{bashtext}{\texttt{\%~~#1}}}}
\newcommand{\troot}[1]{\hspace{10pt}\colorbox{bashbackground}{\textcolor{bashtext}{\texttt{root[~]~~#1}}}}

\setlength{\parindent}{0pt}

\title{Appunti Lezione 05}
\author{Lorenzo Visca}
\date{}

\begin{document}

\maketitle

\section{Soluzione: verifica del teorema del limite centrale}
Codice di riferimento: \ttt{centralSol.C} \vspace{10pt}

Il codice di riferimento implementa alcune migliorie rispetto a quello fatto come esercizio:

\begin{itemize}
    \item La macro prende come argomento extra una stringa, a seconda del valore passatole il programma utilizza una diversa funzione per generare i numeri casuali.
          La funzione \ttt{randomlong}, presentata a titolo di esempio, è un metodo più inefficiente che rifiuta i numeri estratti se non rientrano nell'intervallo desiderato.
          Per scegliere la funzione da utilizzare si definisce il puntatore a funzione \ttt{randFunc} che viene inizializzato in base al valore della stringa passata come argomento.
    \item Si utilizza un seed fisso per il generatore di numeri casuali in modo da poter riprodurre i risultati ottenuti in caso di necessità (debugging).
    \item Viene creato un array di istogrammi per gestirli in modo più efficiente.
    \item Gli istogrammi hanno normalmente un range $5\sigma$, ma sono stati aggiustati per evitare di uscire dai limiti fisici dell'istogramma (0 e N).
    \item Viene utilizzata un'unica somma cumulativa per tutti gli istogrammi, inserendo il valore calcolato nell'istogramma corretto in base al numero di estrazioni corrispondente.
\end{itemize}

\newpage

\newpage
\section{Codici}

\subsection{centralSol.C}
\lstinputlisting[language=C++]{../centralSol.C}


\end{document}