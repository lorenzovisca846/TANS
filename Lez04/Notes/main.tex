\documentclass[10pt]{article}
\usepackage{amsmath}
\usepackage{amssymb}
\usepackage{anysize}
\usepackage{listings}
\usepackage{graphicx}
\usepackage{xcolor}


\definecolor{Blue}{rgb}{0.2,0.2,0.9}
\definecolor{Green}{rgb}{0,0.6,0}
\definecolor{Gray}{rgb}{0.5,0.5,0.5}
\definecolor{Purple}{rgb}{0.58,0,0.82}
\definecolor{background}{rgb}{0.95,0.95,0.92}

\definecolor{bashtext}{rgb}{0.9,0.9,0.9}
\definecolor{bashbackground}{rgb}{0.3,0.3,0.3}

\lstdefinestyle{mystyle}{
    backgroundcolor=\color{background},   
    commentstyle=\color{Green},
    keywordstyle=\color{Blue},
    numberstyle=\tiny\color{Gray},
    stringstyle=\color{Purple},
    basicstyle=\ttfamily\footnotesize,
    breakatwhitespace=false,         
    breaklines=true,                 
    captionpos=b,                    
    keepspaces=true,                 
    numbers=left,                    
    numbersep=5pt,                  
    showspaces=false,                
    showstringspaces=false,
    showtabs=false,                  
    tabsize=2
}
\lstset{style=mystyle}

\newcommand{\ttt}{\texttt}

\newcommand{\tcpp}[1]{\hspace{10pt}\colorbox{background}{\textcolor{black}{\texttt{#1}}}}
\newcommand{\tbash}[1]{\hspace{10pt}\colorbox{bashbackground}{\textcolor{bashtext}{\texttt{\%~~#1}}}}
\newcommand{\troot}[1]{\hspace{10pt}\colorbox{bashbackground}{\textcolor{bashtext}{\texttt{root[~]~~#1}}}}

\setlength{\parindent}{0pt}

\title{Appunti Lezione 04}
\author{Lorenzo Visca}
\date{}

\begin{document}

\maketitle

\section{Controllo aggiuntivo sulla formattazione dei file}
Codice di riferimento: \ttt{ioexample3.C} \vspace{10pt}

Può accadere che un file di dati in input contenga delle stringhe di commento o delle righe vuote.
La libreria \ttt{sstream} permette di gestire questi casi in modo semplice:

\begin{enumerate}
    \item Il file viene letto riga per riga con la funzione \ttt{getline()}.
    \item La riga viene convertita in uno stream di stringa \ttt{iss} con la funzione \ttt{istringstream}: in questo modo ogni riga permette di estrarre i dati.
    \item La sezione di codice che gestisce l'estrazione dei dati viene condizionata a \ttt{(iss >> x)}: in questo modo
          quando si incontra un valore non numerico (come un commento) la condizione non è più rispettata e si passa alla riga successiva.
    
\end{enumerate}

NOTA: se un commento è seguito da altri dati validi nella stessa riga questi ultimi non vengono letti;
tuttavia i file di dati sono generalmente formattati in modo da non avere questo problema,
evitando così controlli ulteriori che rallenterebbero la lettura del file.

\section{Esercizio: verifica del teorema del limite centrale}
Codice di riferimento: \ttt{central.C} \vspace{10pt}

Dato un parametro \( w \in [0, 0.5] \), definiamo la seguente densità di probabilità:

\begin{equation}
f(x) = \begin{cases}
\frac{1}{2w} & \text{if } x \in [0,w]\cup[1-w,1] \\
0 & \text{elsewhere}
\end{cases}
\label{eq:dist_w}
\end{equation}

I momenti di questa distribuzione sono:
\begin{itemize}
    \item \(\displaystyle E[x]=\int_{-\infty}^{\infty} x\,f(x)\,dx=\int_0^w \frac{x}{2w}\,dx+\int_{1-w}^1 \frac{x}{2w}\,dx=\tfrac{1}{2}\)
    \item \(\displaystyle E[x^2]=\int_{-\infty}^{\infty} x^2\,f(x)\,dx=\int_0^w \frac{x^2}{2w}\,dx+\int_{1-w}^1 \frac{x^2}{2w}\,dx=\tfrac{1}{6}(2w^2-3w+3)\)
    \item \(\displaystyle V[x]=E[x^2]-E[x]^2=\tfrac{1}{6}(2w^2-3w+3)-\tfrac{1}{4}\)
\end{itemize}

\vspace{10pt}

L'esercizio consiste nell'estrarre campioni della somma di N valori di x estratti da questa distribuzione, e verificare che la distribuzione della somma tende a una gaussiana al crescere di N (teorema del limite centrale).
Per ogni valore di N si utilizza un campione di \(2 \times 10^6\) estrazioni.

Viene inoltre richiesto di verificare che la media e la varianza della distribuzione della somma siano coerenti con i valori attesi:
\begin{itemize}
    \item \(\displaystyle E[S] = N\cdot E[x] \)
    \item \(\displaystyle V[S] = N\cdot V[x] \)
\end{itemize}


\section{Generazione di numeri casuali}
Codice di riferimento: \ttt{central.C} \vspace{10pt}

le funzioni generatrici di numeri casuali sono di fatto funzioni che estraggono un valore da una lista di numeri pre-calcolati.
In questo senso i numeri generati sono detti pseudo-casuali: l'unica garanzia è data da una distribuzione sufficientemente caotica dei numeri in questa lista, e una lunghezza sufficiente a non avere pattern ripetuti.

Alla prima estrazione viene scelto un punto di partenza nella lista, detto seed, e da questo punto in poi i numeri vengono estratti in sequenza: se il seed è lo stesso, la sequenza di numeri estratti sarà la stessa.

La sintassi per la generazione di numeri casuali in ROOT, gestita dalla libreria \ttt{TRandom3}, è la seguente:

\vspace{10pt}

\tcpp{TRandom3 *myptr = new TRandom3(seed);}

\tcpp{double x = myptr->Rndm();}

\vspace{10pt}

Dove \ttt{Rndm()} estrae un numero uniformemente distribuito nell'intervallo \([0,1]\).

Se si devono generare più sequenze di numeri casuali, inizializzando generatori distinti c'è il rischio che i seed siano troppo vicini e le sequenze risultino quindi correlate.

Per evitare questo problema si può utilizzare un unico generatore globale in modo che tutte le estrazioni vengano fatte sequenzialmente;
in ROOT questo generatore è accessibile tramite il puntatore globale \ttt{gRandom} (in generale i puntatori globali in ROOT iniziano con la lettera \ttt{g}):

\vspace{10pt}

\tcpp{double x = gRandom->Rndm();}

\subsection{Estrazione dalla distribuzione scelta}

Dalla teoria delle distribuzioni di probabilità sappiamo che data una variabile \(x\) con distribuzione \(f(x)\) e distribuzione cumulativa \(F(x)\), la variabile \(r=F(x)\) è distribuita uniformemente nell'intervallo \([0,1]\).

Di conseguenza, estraendo una variabile \(r\) uniformemente distribuita in \([0,1]\) si può campionare una variabile \(x\) da una distribuzione arbitraria calcolando \(x=F^{-1}(r)\), quando la funzione \(F\) è invertibile.

Nel caso della distribuzione definita in \eqref{eq:dist_w}, la funzione cumulativa è:
\begin{equation*}
F(x) = \begin{cases}
    \frac{x}{2w} & \text{se } x \in [0,w] \\
    \frac{1}{2} + \frac{x - (1 - w)}{2w} & \text{se } x \in [1-w,1] \\
\end{cases}
\end{equation*}

La funzione inversa con cui si estrae \(x\) da \(r\) è quindi:
\begin{equation*}
x = F^{-1}(r) =
\begin{cases}
    2wr & \text{se } r \in [0, \tfrac{1}{2}] \\
    1 - 2w(1 - r) & \text{se } r \in [\tfrac{1}{2}, 1] \\
\end{cases}
\end{equation*}

La funzione \ttt{randomeff()} implementa una versione più efficiente di questa estrazione con il cambio di variabile \(r' = 2wr\), in modo da evitare una moltiplicazione risparmiando tempo di calcolo.

\newpage
\section{Codici}

\subsection{ioexample3.C}
\lstinputlisting[language=C++]{../ioexample3.C}

\newpage

\subsection{central.C}
\lstinputlisting[language=C++]{../central.C}

\end{document}