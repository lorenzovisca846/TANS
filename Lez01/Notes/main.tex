\documentclass{article}
\usepackage{amsmath}
\usepackage{amssymb}
\usepackage{anysize}
\usepackage{listings}
\usepackage{graphicx}
\usepackage{xcolor}

\definecolor{Blue}{rgb}{0.2,0.2,0.9}
\definecolor{Green}{rgb}{0,0.6,0}
\definecolor{Gray}{rgb}{0.5,0.5,0.5}
\definecolor{Purple}{rgb}{0.58,0,0.82}
\definecolor{background}{rgb}{0.95,0.95,0.92}

\definecolor{bashtext}{rgb}{0.9,0.9,0.9}
\definecolor{bashbackground}{rgb}{0.3,0.3,0.3}

\lstdefinestyle{mystyle}{
    backgroundcolor=\color{background},   
    commentstyle=\color{Green},
    keywordstyle=\color{Blue},
    numberstyle=\tiny\color{Gray},
    stringstyle=\color{Purple},
    basicstyle=\ttfamily\footnotesize,
    breakatwhitespace=false,         
    breaklines=true,                 
    captionpos=b,                    
    keepspaces=true,                 
    numbers=left,                    
    numbersep=5pt,                  
    showspaces=false,                
    showstringspaces=false,
    showtabs=false,                  
    tabsize=2
}
\lstset{style=mystyle}

\newcommand{\ttt}{\texttt}

\newcommand{\tcpp}[1]{\colorbox{background}{\textcolor{black}{\texttt{#1}}}}
\newcommand{\tbash}[1]{\hspace{10pt}\colorbox{bashbackground}{\textcolor{bashtext}{\texttt{\%~~#1}}}}
\newcommand{\troot}[1]{\hspace{10pt}\colorbox{bashbackground}{\textcolor{bashtext}{\texttt{root[~]~~#1}}}}

\setlength{\parindent}{0pt}

\title{Appunti Lezione 01}
\author{Lorenzo Visca}
\date{}

\begin{document}

\maketitle

\section{Gestione di file input/output}
Codice di riferimento: \ttt{ioexample.C} \vspace{10pt}

La classe \ttt{fstream} permette di gestire file in input e output. I file sono associati alle variabili \ttt{in} e \ttt{fout} alle righe 9 e 16.
Il blocco di codice tra le righe 9 e 21 si occupa dell'apertura dei file con eventuale controllo degli errori.

Per leggere un valore dal file di input si usa il comando \ttt{>>}, che è \textit{overloaded} per la classe \ttt{fstream}:

\hspace{10pt}\tcpp{in >> variabile;}

Per scrivere un valore sul file di output si usa il comando \ttt{<<}, anch'esso \textit{overloaded} per la classe \ttt{fstream}:

\hspace{10pt}\tcpp{fout << variabile;}

Dopo aver finito di usare i file, è buona norma chiuderli con il comando \ttt{close()}.
\vspace{5pt}

Il codice \ttt{ioexample.C} prende in input il nome del file di input e il nome del file di output, dopodiché legge coppie di numeri contenute nel file di input e le trascrive nel file di output.

\section{Utilizzo delle macro di ROOT}
Le macro di ROOT vengono convenzionalmente salvate con estensione \ttt{.C} e possono essere eseguite direttamente nell'interprete di ROOT. Per caricare il file \ttt{ioexample.C} nell'interprete di ROOT, si usa il comando \ttt{.L}, quindi si può eseguire la macro passandole come argomento i nomi dei file di input e di output come richiesto dalla funzione alla riga 7:
\vspace{5pt}

\troot{.L ioexample.C}

\troot{ioexample("input.dat","output.dat")}
\vspace{5pt}

Dove i file \ttt{input.dat} e \ttt{output.dat} si trovano nella stessa cartella della macro. In alternativa, si possono specificare i percorsi assoluti o relativi dei file. Convenzionalmente si usa l'estensione \ttt{.dat} per i file contenenti dati numerici separati da spazi.

\newpage
\section{Codici}

\ttt{ioexample.C}
\lstinputlisting[language=C++]{../ioexample.C}

\end{document}